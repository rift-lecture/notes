\chapter{The \rift Language}

As a running example we will write a complete compiler for a dynamic language called \rift, a minimalistic variation on the R language. \Rift was created for the purpose of this lecture, thus features as few language features as possible while still providing many of the vast challenges a full blown dynamic language would provide to the implementor.

\Rift -- like its inspiration R -- is a vectorized language. As such it does not have any scalar types, \ie the number 1 is represented a numerical vector which happens to be of length one.

As far as primitive types go, the language provides only 3 thereof:
\begin{itemize}
\item Double precision numbers
\item Eight bit characters
\item Functions
\end{itemize}
Of those the first two are vectorized; the function is a first class value in the language but exists only as a scalar.

\section{Tutorial}

The following snippets contain fully functional \rift code intended to provide a gentle introduction how the language is supposed to work.

As stated before there are two basic data types,

\begin{lstlisting}[language=rift]
# First: doubles
1.2
1.0 + 1

# to be precisely vectors of doubles
c(1, 2, 3)

# since even scalars exist as vectors.
1[0]
c(1, c(2, 3))

# Second: vectors of characters represented as strings.
"hi"
c("hi ", "friend")

# We have operations to ask the type of values and their length.
type(v)
type(str)
type(1)
length(diag)
length(1)
length(str)

# And of course variables.
v <- 1
s <- "s"

# But types do not mix, and we don't have casts.
1 + "two"
c(1, "a")
\end{lstlisting}

A third data type is the function. Functions are defined by the $function$ keyword and are expressions themselves. They can be assigned to variables or passed as arguments.

\begin{lstlisting}[language=rift]
# Functions can be created as exppressions and bound to names.
id <- function( x ) { x }
id
id(1)
id("hi")

# Function are higher order.
one <- function() { 1 }
two <- function() { 2 }

ho <- function(a,b) {
  if ( a() > b() ) {
    a()
  } else {
    b()
  }
}
ho( one, two )
ho( two, one )

# Functions are lexically scoped.
builder <- function() {
  x <- 0
  function() {
    x[0] <- x[0] + 1
  }
}
counter <- builder()
counter()
counter()

# Values are passed by reference.
x <- c(1, 2)
add1 <- function ( v ) {
  v[0] <- v[0] + 1
}
add1(x)
x
\end{lstlisting}

Note that functions are not defined with an unique name. They are just assigned to normal variables. To call a function its definition will thus need to be looked up at runtime.

Additionally as an additional source of evilness towards the implementor the language also features a dynamic eval function to run arbitrary code.

\begin{lstlisting}[language=rift]
eval("x <- 1")
f <- function() {
  z <- 42
  eval("x <- z")
}
f()
\end{lstlisting}

